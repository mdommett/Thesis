\chapter{Chemistry on a Computer}
\newcommand{\schro}{Schr\"{o}dinger}
\label{chapter:theory}
\section{Quantum Mechanics and Chemistry}\label{section: theory}
\subsection{Overview}\label{section: theory_overview}
The development of quantum mechanics in the early 20th century armed scientists with the tools to calculate the microscopic properties of matter. In chemistry, the postulates of quantum mechanics can be applied to calculate relative energies of molecules, molecular geometries, ratios of products of chemical reactions, transition states, spectra, and any other phenomenon of interest. However, whilst in principle any property can be calculated exactly by the \schro{} equation, the analytical solution is only obtainable for systems with one electron. For systems larger than this, and therefore anything of observable chemical relevance, the computational expense on even modern computer architecture is intractable. 

To overcome this, a number of approximations are used field of computational chemistry. In general, as the number of atoms one wants to model increases, qualitative nature of the result also increases. Computational chemistry methods can generally be split into the types of approximations made and the number of atoms the method wishes to treat. In biophysical process and the modelling of proteins on the scale of tens of thousands of atoms, quantum mechanics (and electrons) are ignored completely. Forcefields are used to calculate the energy corresponding to a set of atomic coordinates in what are known as the \ac{MM} class of methods. The interactions between atoms are defined by analytical potentials, such as for bond stretches, bends, and angles, and are parameterised for different types of molecules using more accurate methods. This procedure is time-consuming and makes the forcefield specific to the systems it was fitted to, but enables computationally facile access to molecular geometries and properties of large systems.

At the other end of the scale, for systems of typically less than 500-1000 atoms, the electronic structure is included through wavefunction and \ac{DFT} techniques. For the applications involved in this thesis, involving photoinduced phenomena, the activity of the electrons is paramount, and as such it is these methods which are utilised herein. In the next sections, the importance of the \schro{} equation shall be established, which along with the Born-Oppenheimer approximation enables the calculation of electronic properties through the simplest wavefunction method, the \ac{HF} method. 
\subsection{The \schro{} Equation}\label{section: theory_schrodinger}